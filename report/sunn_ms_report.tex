\documentclass{article}
\usepackage{amsmath}
\usepackage{graphicx}
\usepackage[backend=biber, bibencoding=utf8]{biblatex}
\usepackage[margin=1in]{geometry}
\graphicspath{{./images/}}
\addbibresource{ref.bib}

\title{Comparison of Gaussian copula and random forest in zero-inflated spatial prediction}
\date{\today}
\author{Nick Sun}

\begin{document}

\maketitle

\begin{abstract}
	Spatial prediciton and interpolation arises in many fields like blah blah blah.
	The random forest is blah blah blah.
	The Gaussian copula is an extension of the spatial linear model that blah blah blah.
	One sentence summary of simulation study results.
\end{abstract}

\section{Introduction}
Here is my introduction where I talk about the two methods: random forests\cite{rfsp} and Gaussian copula \cite{madsen09}

\section{Data}
The forestry inventory data used here was made available by the Forestry Inventory and Analysis program of the USDA Forest Service.
It contains inventory information on 13 variables of interest across 1224 plots of land in NW Oregon.
The response variables of interest include total volume, total biomass, total number of trees, and volume of specific tree species.
The dataset includes fuzzed latitude, longitude, and elevation information.
The possible covariate variables include annual precipitation, tc3\cite{raynolds16} wetness index, annual temperature, NDVI, and cover.

\begin{figure}[ht]
\includegraphics[scale=.7]{total_timber_map}
\centering
\end{figure}

The response variables are positively skewed and zero inflated.

\begin{figure}[ht]
	\includegraphics[scale=.4]{raw_histograms}
	\centering
	\caption{Histograms of raw response variables}
\end{figure}

\section{Methods}
Here is my methods section where I talk about the math behind random forests and Gaussian copula\cite{verhoef02}.

\section{Results}
Here is my results section where I talk about the outcome of my simulation studies

\section{Conclusion}
Here is a conclusion where I will probably say something like Gaussian copula is better because statistics over machine learning 5ever.

\printbibliography

\end{document}
